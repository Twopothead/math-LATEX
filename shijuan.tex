% !Mode::"TeX:UTF-8"
\documentclass[cs4size,UTF8]{ctexart}
\usepackage{lastpage}
\usepackage{fancyhdr}
\usepackage{setspace} % 行间距
\usepackage{amsmath,amsfonts,amsthm,graphicx,amssymb,enumerate}
\usepackage{txfonts}
%\usepackage{tikz,pgfplots} %绘图
%\usepackage{tkz-euclide,pgfplots}
%\usetikzlibrary{automata,positioning}
\usepackage[paperwidth=21cm,paperheight=29.7cm,top=2cm,bottom=2.5cm,left=2.3cm,right=2cm]{geometry} % 单页


%选择题
\newcommand{\fourch}[4]{\\\begin{tabular}{*{4}{@{}p{3.5cm}}}(A)~#1 & (B)~#2 & (C)~#3 & (D)~#4\end{tabular}} % 四行
\newcommand{\twoch}[4]{\\\begin{tabular}{*{2}{@{}p{7cm}}}(A)~#1 & (B)~#2\end{tabular}\\\begin{tabular}{*{2}{@{}p{7cm}}}(C)~#3 &
		(D)~#4\end{tabular}}  %两行
\newcommand{\onech}[4]{\\(A)~#1 \\ (B)~#2 \\ (C)~#3 \\ (D)~#4}  % 一行

\renewcommand{\headrulewidth}{0pt}
\pagestyle{fancy}
\begin{document} % 在begin前面加了一个空格以免出现显示错误,编译时应该去掉
\fancyhf{}
\fancyfoot[CO,CE]{\vspace*{1mm}第\,~\thepage\,~页 , 共 ~\pageref{LastPage} 页}
	
\begin{spacing}{1.25}
	\begin{center}
\begin{LARGE}
\heiti{~$2018~-2019$学年第一学期期末考试\\
高一年级数学试卷\\}
\end{LARGE}
\end{center}
\end{spacing}
\vspace{0.2cm}

\begin{center}
班级\underline{~~~~~~~~~~~~~~~~~~~}姓名\underline{~~~~~~~~~~~~~~~~~~~}得分\underline{~~~~~~~~~~~~~~~~~~~}
\end{center}

\vspace{-2cm}
\begin{spacing}{1.5}		
\section*{
\begin{flushleft}
\kaishu{一、选择题~(每题~5 分,共~60 分)}
\end{flushleft}
}
\vspace{-0.5cm}		
\begin{enumerate}\setcounter{enumi}{0}
\item 与~$60^\circ $角终边相同的角的集合是 ~(~~~~~~~~)
\onech{\{$\alpha\mid \alpha=k\bullet360^\circ+\dfrac{\pi}{3},k\in Z$\}.}{\{$\alpha\mid \alpha=2k\pi+60^\circ,k\in Z$\}.}{\{$\alpha\mid \alpha=2k\bullet360^\circ+60^\circ,k\in Z$\}.}{\{$\alpha\mid \alpha=2k\pi+\dfrac{\pi}{3},k\in Z$\}.}
\item $\sin ~585^\circ$的值为(~~~~~~~~)
\fourch{$-\dfrac{\sqrt{2}}{2}$.}{$\dfrac{\sqrt{2}}{2}$.}{ $-\dfrac{\sqrt{3}}{2}$ .}{$\dfrac{\sqrt{3}}{2}$.}
			
\item 函数~$y=2\tan(\dfrac{1}{2}x-\dfrac{\pi}{4})$ 的最小正周期是~(~~~~~~~~)
\fourch{$~\pi$.}{$~2\pi$.}{$~3\pi$.}{$~4\pi$.}
			
\item 已知~$\log_x{ 16} = 2$,  则~$x= $ ~(~~~~~~~~)
\fourch{$\pm{4}$.}{$4$.}{$256$.}{$2$.}
\item 已知集合~$U=\{1,2,3,4,5,6,7\}$ , ~$A=\{2,4,5,7\}$ , ~$B=\{3,4,5\}$,则$(\complement_u A)\cap(\complement_u B)=$ ~(~~~~~~~~)
 \fourch{~$\{1,6\}$.}{~$\{4,5\}$.}{~$\{2,3,4,5,7\}$.}{~$\{1,2,3,6,7\}$.}
\item 若函数~$f(x)=2x+1$ , ~则$f(f(x))=$ ~(~~~~~~~~)
 \fourch{~$4x+3$.}{~$4x+4$.}{~$(2x+1)^2$.}{~$2x^2+2$.}
\item 已知~$A(x,2)$ , ~$B=(5,y-2)$ , ~若$\overrightarrow{AB}=(4,6)$,则$x,y$的值分别是 ~(~~~~~~~~)
 \fourch{~$x=-1,y=10$.}{~$x=1,y=10$.}{~$x=1,y=-10$.}{~$x=-1,y=-10$.}
 \item 若~$A=(3,6)$ , ~$B=(-5,2)$ , ~$C=(6,y)$三点共线,则$y$的值为 ~(~~~~~~~~)
 \fourch{~$13$.}{~$-13$.}{~$9$.}{~$-9$.}
 \item 下列等式一定能成立的是 ~(~~~~~~~~)
 \twoch{~$\overrightarrow{AB}+\overrightarrow{AC}=\overrightarrow{BC}$.}{~$\overrightarrow{AB}-\overrightarrow{AC}=\overrightarrow{BC}$.}
 {~$\overrightarrow{AB}+\overrightarrow{AC}=\overrightarrow{CB}$.}{~$\overrightarrow{AB}-\overrightarrow{AC}=\overrightarrow{CB}$.}
 \item 下列函数中同时满足: \textcircled{\footnotesize{1}}在~$(0,\dfrac{\pi}{2})$上是增函数 ;\textcircled{\footnotesize{2}} 奇函数;\textcircled{\footnotesize{3}} 以 ~$\pi$ 为最小正周期的函数的是~(~~~~~~~~)
 \fourch{~$y=\tan{x}$.}{~$y=\cos{x}$.}{~$y=\sin{x}$.}{~$y=|\sin{x}|$.}
\item 设~$\overrightarrow{e_1}$ 与 ~$\overrightarrow{e_2}$ 是平面内的一组基底,则下列四组向量中,不能作为基底的是~(~~~~~~~~)
 \twoch{~$\overrightarrow{e_1}+\overrightarrow{e_2}$和$
 \overrightarrow{e_1}-\overrightarrow{e_2}$.}{~$3\overrightarrow{e_1}-2\overrightarrow{e_2}$和$-6\overrightarrow{e_1}+4\overrightarrow{e_2}$.}{~$\overrightarrow{e_1}+2\overrightarrow{e_2}$和$\overrightarrow{e_2}+2\overrightarrow{e_1}$.} {~$\overrightarrow{e_2}$和$\overrightarrow{e_1}+\overrightarrow{e_2}$.}
\item 若函数~$f(x)=2\sin(\omega x+\varphi)$ ,$x\in R$(其中~$\omega>0,|\varphi|<\dfrac{\pi}{2}$)的最小正周期为$\pi$ , 且~$f(0)=\sqrt{3}$, 则~(~~~~~~~~)
 \fourch{~$\omega=\dfrac{1}{2},\varphi=\dfrac{\pi}{6}$.}{~$\omega=\dfrac{1}{2},\varphi=\dfrac{\pi}{3}$.}
 {~$\omega=2,\varphi=\dfrac{\pi}{6}$.}{~$\omega=2,\varphi=\dfrac{\pi}{3}$.}
\end{enumerate}
\vspace{-1.5cm}

\section*{
    \begin{flushleft}
 \kaishu{二、填空题~(每题~5 分, 共~ 20 分)}
    \end{flushleft}
}
\vspace{0cm}

\begin{enumerate}\setcounter{enumi}{12}
\item 已知向量~$\overrightarrow{a}=(x-5,3),\overrightarrow{b}=(2,x)$ ,且
$\overrightarrow{a} \bot \overrightarrow{b}$,则~$x$的值为\underline{~~~~~~~~~~~~~}.
			
\item 函数~$f(x)=\sqrt{2x+3}+\dfrac{1}{x-1}$ 的定义域~\underline{~~~~~~~~~~~~~~}.
			
\item 已知~$|\overrightarrow{a}|=1,|\overrightarrow{b}|=2,\overrightarrow{a} \bot \overrightarrow{b}$, 则 ~$|\overrightarrow{a}+\overrightarrow{b}|=$~\underline{~~~~~~~~~~~~~~}.
\item ~$f(x)$是偶函数,当$x<0$时,$f(x)=x(x+1)$,则$f(-2)=$~\underline{~~~~~~~~~~~~~~}.
			
		\end{enumerate}
\vspace{-1.5cm}				
\section*{
   \begin{flushleft}
 \kaishu{三、解答题~(请写出必要的文字说明、解题过程等,共70分)}
   \end{flushleft}	
}
\begin{enumerate}\setcounter{enumi}{16}
\item ~$(1)$ 将下列角度与弧度互化.\\
$~~~~~~~~~~~~\textcircled{\footnotesize{1}} ~\dfrac{8}{5}\cdot \pi$;~~~~~~~ ~$\textcircled{\footnotesize{2}} ~1020^\circ$.\\
~~$~~(2)$ 已知角 ~$\alpha$~~终边经过点~$(-5,12)~$, 求它的正弦、余弦、正切值.		
\vspace{6cm}

\item  已知~$\tan(\pi+\alpha)=3$ ,求下列各式的值.\\
$~~~~~~\textcircled{\footnotesize{1}} ~\dfrac{4\sin{\alpha}-\cos{\alpha}}{3\sin{\alpha}+5\cos{\alpha}}$;~~~~~~~ ~$\textcircled{\footnotesize{2}} ~\dfrac{\sin^2{\alpha}-2\sin{\alpha} \cos{\alpha}-\cos^2{\alpha}}{4\cos^2{\alpha}-3\sin^2{\alpha}}$.	
\vspace{6cm}


\item 已知函数~$y=2\sin(x-\dfrac{\pi}{4})$.\\
 $~\textcircled{\footnotesize{1}}$求该函数的单调递增区间;\\~~~~~~~~$\textcircled{\footnotesize{2}}$ 由$y=\sin{x}$ 的图象如何变换得到该函数的图象?			
\vspace{6cm}			


\item 已知$\alpha$是第三象限角,且~$f(\alpha)=\dfrac{\sin(\pi-\alpha) \cos(2\pi-\alpha) \tan(-\alpha+\dfrac{3\pi}{2})}{\cot(-\alpha-\pi) \sin(-\alpha-\pi)}$.\\ 
$~\textcircled{\footnotesize{1}}$ 化简~$f(\alpha).$\\
$~\textcircled{\footnotesize{2}}$ 若~$\cos(\alpha-\dfrac{3\pi}{2})=\dfrac{1}{5}$,求~$f(\alpha)$.
\vspace{6cm}


\item 已知~$f(x)=\dfrac{ax+b}{1+x^2}$是定义在~$(-1,1)$上的奇函数,且$~f(1)=1$.\\
$~\textcircled{\footnotesize{1}}$确定函数$~f(x)$的解析式;
$~\textcircled{\footnotesize{2}}$用定义证明$~f(x)$在$~(-1,1)$上是增函数.
\vspace{10cm}			


\item 平面内给定三个向量~$\overrightarrow{a}=(3,2),\overrightarrow{b}=(-1,2),\overrightarrow{c}=(4,1)$.\\
$~\textcircled{\footnotesize{1}}$求满足$~\overrightarrow{a}=m\overrightarrow{b}+n\overrightarrow{c}$的实数$~m,n$;\\
$~\textcircled{\footnotesize{2}}$若$~(\overrightarrow{a}+k\overrightarrow{c})\parallel(2\overrightarrow{b}-\overrightarrow{a})$,求实数$~k$;\\
$~\textcircled{\footnotesize{3}}$设$~\overrightarrow{d}=(x,y)$,满足$~(\overrightarrow{d}-\overrightarrow{c})\parallel (\overrightarrow{a}+\overrightarrow{b})$且$~|\overrightarrow{d}-\overrightarrow{c}|=1$,求$~\overrightarrow{d}$.			


\end{enumerate}
\end{spacing}
\clearpage
	
\end{document}
