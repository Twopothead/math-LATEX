\documentclass{BHCexam}
\biaoti{~$2016 - 2017$~学年度期中考试}
\fubiaoti{高二数学试卷}
\usepackage{palatino}
\usepackage{siunitx}%输入度数符号需要的单位宏包
\usepackage{tikz}
\usetikzlibrary{shapes.geometric, arrows}
\tikzstyle{startstop} = [rectangle, rounded corners, minimum width = 1cm, minimum height=0.5cm,text centered, draw = black]
\tikzstyle{io} = [trapezium, trapezium left angle=70, trapezium right angle=110, minimum width=0.5cm, minimum height=0.5cm, text centered, draw=black]
\tikzstyle{process} = [rectangle, minimum width=2cm, minimum height=0.5cm, text centered, draw=black]
\tikzstyle{decision} = [diamond, aspect = 2, text centered, draw=black]
% 箭头形式
\tikzstyle{arrow} = [->,>=latex]
\zihao{-4}

\begin{document}
\maketitle
\mininotice

\begin{questions}

%选择题
\xuanze
\qs 程序框图符号“~%
\begin{tikzpicture}
\draw (0,0) rectangle (0.7,0.4);
\end{tikzpicture}
”可用于\tk.\\%
\onech{$\text{输出}a=10$}{$\text{赋值}a=10$}{$\text{判断}a=10$}{$\text{输入}a=10$}
\qs 阅读下面的程序框图,该程序输出的结果是\tk.\\
\onech{$12$}{$19$}{$28$}{$37$}
\begin{flushright}
\begin{tikzpicture}[node distance=2pt,scale=0.5]
%定义流程图具体形状
\node[startstop](start){开始};
\node[process, below of = start, yshift = -1cm](in1){$a=1,S=1$};
%\node[process, below of = in1, yshift = -1cm](pro1){$S=\frac{1}{1-S}$};
\node[decision, below of = in1, yshift = -1cm](dec1){ $a>3?$};
\node[process, below of = dec1, yshift = -1cm](pro2){$S=S+9$};
\node[process, below of = pro2, yshift = -1cm](pro3){$a=a+1$};
\node[io, below of = pro3, yshift = -1cm](out1){输出~$S$~};
\node[startstop, below of = out1, yshift = -1cm](stop){结束};
\coordinate (point1) at (0cm, -5cm);
\coordinate (point2) at (1cm,-2cm);
%连接具体形状
\draw [arrow] (start) -- (in1);
\draw [arrow] (in1) -- (dec1);
\draw [arrow] (dec1) -- node[right]{\small 否}(pro2);
\draw [arrow] (pro2)-- (pro3);
%\draw [arrow] (point2) -- (point1);
%\draw [arrow] (dec1) -- node [left] {是} (out1);
%\draw [arrow] (dec1) -- node [right] {是} (out1);
\draw [arrow] (out1) -- (stop);
\draw [arrow] (2.2,-4.3) --(2.5,-4.3)--(2.5,-9.5)--(0,-9.5)--(0,-10.1);
\draw [<-] (0,-3)--(-2.4,-3)--(-2.4,-8.5)--(-2,-8.5);
\node [right] at (2.5,-7.2) {\small 是};
\end{tikzpicture}
\end{flushright}
\vspace{-8cm}
\qs 在下列各数中,最大的数是\tk.\\
\twoch{$85_{(9)}$}{$210_{(6)}$}{$1000_{(4)}$}{$11111_{(2)}$}
\qs 不在$ 3x+2y<6 $表示的平面区域内的一个点是\tk.\\
\twoch{$(0,0)$}{$(1,1)$}{$(0,2)$}{$(2,0)$}
\qs 若 $\Bigg\{\begin{aligned}
&x\leq 2 \\
&y\leq 2,x+y\geq 2
\end{aligned}$,则目标函数$z=x+2y$的取值范围是\tk.\\
\twoch{$\left[2,6\right]$}{$\left[2,5\right]$}{$\left[3,6\right]$}{$\left[3,5\right]$}
\qs 方程$ x^2+y^2+2ax-by+c=0 $表示圆心为$ C(2,2) $,半径为$ 2 $的圆,则$ a,b,c $的值依次为\tk.\\
\onech{$ 2,4,4 $}{$ -2,4,4 $}{$ 2,-4,4 $}{$ 2,-4,-4 $}
\qs 直线$ 3x-4y-4=0 $被圆$ (x-3)^2+y^2=9$截得的弦长为\tk.\\
\onech{$ 2\sqrt{2} $}{$ 4 $}{$ 4\sqrt{2} $}{$ 2 $}
\qs 我校高中生共有$ 2700 $人,其中高一年级$ 900 $人,高二年级$ 1200 $人,高三年级$ 600 $人,现采取分层抽样法抽取容量为$ 135 $的样本,则高一、高二、高三各年级抽取的人数分别为\tk.\\
\onech{$ 45,75,15 $}{$ 45,45,45 $}{$ 30,90,15 $}{$ 45,60,30 $}
\qs $ 200 $辆汽车经过某一雷达地区,时速频率分布直方图如\\%
右图所示,则时速超过$ 70~km/h $的汽车数量为\tk.\\
\fourch{$ 2~\text{辆}$}{$ 10~\text{辆}$}{$ 20~\text{辆}$}{$ 70~\text{辆}$}
\vspace{-5cm}
\begin{flushright}
\begin{tikzpicture}[scale=0.8]
\draw[->,>=latex] (0,-0.2)--(0,3.8) node[right] at(0,3.8) {$\dfrac{\text{频率}}{\text{组据}}$};
\draw[->,>=latex] (-0.2,0)--(0.1,0)--(0.2,0.2)--(0.3,-0.2)--(0.5,0)--(6,0) node[below] at(6,0) {时速};
\draw (1,0)--(1,1)--(1.8,1) (1.8,0)--(1.8,2)--(2.6,2) (2.6,0)--(2.6,3)--(3.4,3)--(3.4,0);
\draw (3.4,2.3)--(4.2,2.3)--(4.2,0) (4.2,1.5)--(5,1.5)--(5,0);
\draw[dashed] (0,1)--(1,1) (0,1.5)--(4.2,1.5) (0,2)--(1.8,2) (0,2.3)--(3.4,2.3) (0,3)--(2.6,3);
\node[below] at (1,0) {\small $30$};
\node[below] at (1.8,0) {\small $40$};
\node[below] at (2.6,0) {\small $50$};
\node[below] at (3.4,0) {\small $60$};
\node[below] at (4.2,0) {\small $70$};
\node[below] at (5,0) {\small $80$};
\node[left] at (0,1) {\small $0.005$};
\node[left] at (0,1.5) {\small $0.010$};
\node[left] at (0,2) {\small $0.018$};
\node[left] at (0,2.4) {\small $0.028$};
\node[left] at (0,3) {\small $0.039$};

\end{tikzpicture}
\end{flushright}
\qs 设$ A(3,3,1),B(1,0,5),C(0,1,0) $,则$ AB $的中点$ M $到点$ C $的距离$ \left| ~CM~ \right| $=\tk.\\
\onech{$\dfrac{\sqrt{53}}{4}$}{$\dfrac{53}{2}$}{$\dfrac{\sqrt{53}}{2}$}{$\dfrac{\sqrt{13}}{2}$}
\qs 圆$C_1:~x^2+y^2+2x+8y-8=0$与圆$C_2:~x^2+y^2-4x+4y-2=0$的位置关系是\tk.\\
\onech{相交}{外切}{内切}{相离}
\qs 如图,表示甲、乙两名运动员每场比赛得分情况的茎叶图,则甲和乙得分的中位数的和是\tk.\\
\fourch{$56~\text{分}$}{$57~\text{分}$}{$58~\text{分}$}{$59~\text{分}$}
\vspace{-4cm}
\begin{flushright}
\begin{tabular}{r|c|l}
甲&    &乙 \\
\hline
4&0&8  \\
4~~4&1&2~~5~~8  \\
5~~4&2&3~~6~~5  \\
9~~5~~6~~6~~2~~1&3&2~~3~~4  \\
9~~5&4&1
\end{tabular}
\end{flushright}
%填空题
\tiankong
\qs 把$~98~$化成五进制数为\ltk.
\qs 用辗转相除法求出$153$和$119$的最大公约数是\ltk.
\qs 以点$A(1,4),B(3,-2)$为直径的两个端点的圆的方程为\ltk.
\qs 已知$x,y$满足约束条件%
$\begin{cases}
x-y+5\geq 0 \\
x+y\geq 0  \\
x\leq 3  \\
\end{cases}$,
则$z=4x-y$的最小值为\ltk.

%解答题
\jianda
\qs 已知圆心在直线~$y=0$~上,且圆过两点~$A(1,4),B(3,2)$,求圆的方程.
\vspace{8cm}
\qs 对自行车运动员甲、乙两人在相同条件下进行了$6$次测试,测得他们的最大速度($m/s$)的数据如下:
\begin{center}
\begin{tabular}{|c|c|c|c|c|c|c|}
\hline
~~~~~甲~~~~&$~~~~27~~~~$&$~~~~38~~~~$&$~~~~30~~~~$&$~~~~37~~~~$&$~~~~35~~~~$&$~~~~31~~~~$  \\
\hline
乙&$33$&$29$&$38$&$34$&$28$&$36$  \\
\hline
\end{tabular}
\end{center}
\begin{parts}
\part 画出茎叶图,求中位数;
\part 分别求出甲乙两名自行车赛手最大速度($m/s$)数据的平均数、方差,试判断选谁参加该项重大比赛更合适.
\end{parts}
\vspace{8cm}
\qs 某工厂对产品的产量与成本的资料分析后有如下数据:
\begin{center}
\begin{tabular}{|c|c|c|c|c|}
\hline
~~~~~产量$x$千件~~~~~&~~~~~2~~~~~&~~~~~3~~~~~&~~~~~5~~~~~&~~~~~6~~~~~  \\
\hline
成本$y$万元&7&8&9&12  \\
\hline
\end{tabular}
\end{center}
\begin{parts}
\part  画出散点图;
\part 求成本$y$与产量$x$之间的线性回归方程(结果保留两位小数)
\end{parts}
({\kaishu 参考公式:}$\hat{b}=\dfrac{\sum\limits _{i=1}^{n} x_iy_i-n\overline{x}\overline{y}}{\sum\limits _{i=1}^{n} x_i ^2-n\overline{x}^2},\hat{a}=\overline{y}-\hat{b}\overline{x}$)
\vspace{8cm}
\qs 过原点~$O$~作圆~$x^2+y^2-8x=0$~的弦。
\begin{parts}
\part 求弦~$OA$~中点~$M$~的轨迹方程;
\part 延长~$OA$~到~$N$~,使~$\abs{OA}=\abs{AN}$~,求~$N$~点的轨迹方程。
\end{parts}
\vspace{8cm}
\qs 某中学对高三年级进行身高统计,测量随机抽取的~$40$~名学生的身高,其结果如下(单位:$cm$)\\
{\zihao{-5}%
\begin{tabular}{|c|c|c|c|c|c|c|c|c|c|}
\hline
分组&$\left[140,145\right)$&$\left[145,150\right)$&$\left[150,155\right)$&$\left[155,160\right)$&$\left[160,165\right)$&$\left[165,170\right)$%
&$\left[170,175\right)$&$\left[175,180\right]$&合计 \\
\hline
人数&1&2&5&9&13&6&3&1&40 \\
\hline
\end{tabular}
}
\begin{parts}
\part 列出频率分布表;
\part 画出频率分布直方图;
\part 根据频率直方图求众数、中位数。
\end{parts}
\end{questions}
\end{document}
