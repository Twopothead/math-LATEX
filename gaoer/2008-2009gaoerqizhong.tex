\documentclass{BHCexam}
\biaoti{~$2008 - 2009$~学年度第一学期期中考试}
\fubiaoti{高二数学试卷}


\begin{document}
\maketitle
%\mininotice
\notice
\begin{questions}

%选择题
\xuanze
\question 命题:“~$\exists$~数列~$\{a_n \}$~既是等差数列又是等比数列”是\xx.
\twoch{特称命题并且是真命题}{全称命题并且是假命题}{特称命题并且是假命题}{全称命题并且是真命题}
\question 椭圆~$2x^2+3y^2=12$~的两焦点之间的距离为\xx.
\onech{$2\sqrt{10}$}{$\sqrt{10}$}{$2\sqrt{2}$}{$\sqrt{2}$}
\question 设~$p,q$~是两个简单命题,若“~$\neg p$~且~$\neg q$~”的是真命题,则必有\xx.
\onech{$p$真$q$真}{$p$假$q$假}{$p$真$q$假}{$p$假$q$真}
\question 如果方程~$x^2+ky^2=2$~表示焦点在~$y$~轴上的椭圆,那么实数~$k$~的取值范围是\xx.
\onech{$(0,+\infty)$}{$(0,2)$}{$(1,+\infty)$}{$(0,1)$}
\question 设等比数列~$\{a_n \}$~的公比~$q=2$~,前~$n$~项和为~$S_n$~,则~$\dfrac{S_4}{a_2}=$~\xx.
\onech{$2$}{$4$}{$\dfrac{15}{2}$}{$\dfrac{17}{2}$}
\question 记等差数列的前~$n$~项和为~$S_n$~,若~$S_2=4,S_4=20$~,则该数列的公差~$d=$~\xx.
\onech{$2$}{$3$}{$6$}{$7$}
\question 在三角形~$ABC$~中,~$AB=5,AC=3,BC=7$~,则~$\angle BAC$~的大小为\xx.
\onech{$\dfrac{2\pi}{3}$}{$\dfrac{5\pi}{6}$}{$\dfrac{3\pi}{4}$}{$\dfrac{\pi}{3}$}
\question 若椭圆的短轴为~$AB$~,它的一个焦点为~$F_1$~,则满足~$\triangle ABF_1$~为等边三角形的椭圆的离心率是\xx.
\onech{$\dfrac{1}{4}$}{$\dfrac{1}{2}$}{$\dfrac{\sqrt{2}}{2}$}{$\dfrac{\sqrt{3}}{2}$}
\question 中心在原点,焦点在~$x$~轴上,焦距等于~$6$~,离心率等于~$\dfrac{3}{5}$~,则椭圆的方程是\xx.
\onech{$\dfrac{x^2}{100}+\dfrac{y^2}{36}=1$}{$\dfrac{x^2}{100}+\dfrac{y^2}{64}=1$}%
{$\dfrac{x^2}{25}+\dfrac{y^2}{16}=1$}{$\dfrac{x^2}{25}+\dfrac{y^2}{9}=1$}
\question 若条件~$p:\abs{x+1}\leq 4$~,条件~$q:x^2<5x-6$~,则~$\neg p$~是~$\neg q$~的\xx.
\twoch{充分条件,但不是必要条件}{必要条件,但不是充分条件}{充要条件}{既不是充分条件也不是必要条件}
\question 给出下列四个命题:\\
\ding{192} 若~$a_n=n$~,则数列~$\{a_n \}$~为单调递增数列;\qquad   \ding{193} 若~$a^2>b^2$~,则~$\abs{a}>b$~;\\
\ding{194}椭圆~$\dfrac{x^2}{4}+\dfrac{y^2}{3}=1$~的离心率为~$\dfrac{\sqrt{3}}{2}$~;\quad  \ding{195} 在~$\triangle ABC$~中,若~$sinA=sinB$~,则~$\angle A=\angle B$~.\\
其中真命题有\xx.
\onech{\ding{193}\ding{194}\ding{195}}{\ding{192}\ding{195}}{\ding{192}\ding{193}\ding{195}}{\ding{192}\ding{194}}
\question 已知数列~$\{a_n \}$~满足~$a_1=a,a_2=b,a_{n+2}=a_{n+1}-a_n (n\in N^*)$~.$S_n$~是~$\{a_n \}$~的前~$n$~项和,则~$a_{2010}$~等于\xx.
\onech{$a+b$}{$a-b$}{$-a+b$}{$-a-b$}
%填空题
\tiankong
\question 不等式~$2^{x^2-5}\leq \dfrac{1}{2}$~的解构成的集合为\mtk{}.
\question 已知实数~$x,y$~满足~$ \begin{cases}
x-y+1\geq 0\\
x+y\geq 0 \\
x\leq 0,
\end{cases}$,若~$z=x+2y$~,则~$z$~的范围是\mtk{}.
\question 在~$\triangle ABC$~中,~$B(-2,0),C(2,0),A(x,y)$~给出~$\triangle ABC$~满足的条件,就能得到动点~$A$~的轨迹方程,下面给出了一些条件及方程,请你用线段把左边~$\triangle ABC$~满足的条件及相应的右边~$A$~点的规矩方程连起来(错一条连线不得分).
\begin{center}
\fbox{~~\ding{192} $\triangle ABC$~的周长为~$10$}   \hspace{3cm}   \fbox{ \ding{195} $y^2=25$~~~~~~~~~~~~~~~~~~~~}\\
\vspace{0.5cm}
\fbox{~~\ding{193} $\triangle ABC$~的面积为~$10$}   \hspace{3cm}   \fbox{ \ding{196} $x^2+y^2=4(y\neq 0)$~~}\\
\vspace{0.5cm}
\fbox{~~\ding{194} $\triangle ABC$~中~$\angle A=90^ {\circ}$}   \hspace{3cm}   \fbox{ \ding{197} $\dfrac{x^2}{9}+\dfrac{y^2}{5}=1(y\neq 0)$}
\end{center}
\question 若椭圆~$\dfrac{x^2}{4}+\dfrac{y^2}{b^2}=1(0<b<2)$~的左、右焦点分别为~$F_1,F_2$~是短轴的一个端点,则~$\triangle F_1BF_2$~的面积的最大值是\mtk{}.
%解答题
\jianda
\question 已知~$\{a_n \}$~是一个等差数列,且~$a_2=1,a_5=-5$~.
\begin{parts}
\part 求~$\{a_n \}$~的通项~$a_n$~;
\part 求~$\{a_n \}$~前~$n$~项和~$S_n$~的最大值.
\end{parts}
\vspace{8cm}
\question (12分)如图,~$\triangle ACD$~是等边三角形,~$\triangle ABC$~是等腰直角三角形,~$\angle ACB=90^ {\circ}$~,若~$AB=2$~.
\begin{parts}
\part 求~$\cos {\angle CBD}$~的值;\qquad  \part 求~$BD$~的长.
\end{parts}
\begin{flushright}
\begin{tikzpicture}
\tikzmath{
\s=2*sqrt(2)*cos(105);
\t=2*sqrt(2)*sin(105);
}
\coordinate[label=below left:$A$](a) at (0,0);
\coordinate[label=right:$B$](b) at (4,0);
\coordinate[label=above:$C$](c) at (2,2);
\coordinate[label=below left:$D$](d) at (\s,\t);
\draw(a)--(b)--(c)--(d)--cycle (a)--(c) (b)--(d);
\coordinate[label=below:$E$](e) at (intersection of a--c and b--d);
\end{tikzpicture}
\end{flushright}
\vspace{3cm}
\question 设~$x>0,y>0$~,且~$x+4y=1$~,求~$\dfrac{1}{x}+\dfrac{1}{y}$~的最小值并求出此时~$x$~和~$y$~的值.
\vspace{6cm}
\question (12分)若命题~$r(x):\sin x+\cos x>m,s(x):x^2+mx+1>0$~,如果对于~$\forall x\in R$~,命题~$r(x)$~和~$s(x)$~均为真命题,求实数~$m$~的取值范围.
\vspace{8cm}
\question (12分)设~$F_1,F_2$~为椭圆~$\dfrac{x^2}{9}+\dfrac{y^2}{4}=1$~的两个焦点,~$P$~为椭圆上的一点,若~$\angle F_1PF_2$~为直角,且~$|PF_1|>|PF_2|$~,求~$\dfrac{|PF_1|}{|PF_2|}$~的值。
\end{questions}
\end{document}
